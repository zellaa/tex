%%%%%%%%%%%%%%%%%%%%%%%%%%%%%%%%%%%%%%%%%%%%%%%%%%%%%%%%%%%%%%%%%%%%%%%%%%%%%%
%%%%%%%%%%%%%%%%   Viewgraph Format for QM lectures   %%%%%%%%%%%%%%%%%%%%%%%%
%
%
\count99=1 % This number should be 1 for overheads, 2 for handouts (2 per page)
\ifodd\count99
   \write16{Viewgraph format: 1 per page A4 landscape}
   \magnification\magstep3
   %\special{papersize=297mm,210mm}
   \special{papersize=297mm,410mm}
   \vsize=104mm
   \voffset=-19truemm
\else
   \write16{Viewgraph format: 2 per page A4 portrait}
   \magnification\magstep1
   \special{papersize=210mm,297mm}
   \vsize=215mm
   \voffset=-13truemm
\fi
%\input epsf
\font\oneb=  cmbx12 			% for lecture series (approx 21pt)
\font\ss=    cmssdc10			% for 4-vectors
\font\sss= cmssdc7
\font\ssss= cmssdc5
\textfont12=\ss
\scriptfont12=\sss
\scriptscriptfont12=\ssss
\font\small= cmr8			% footline        (approx 14pt)
\hsize=145mm			% Limit the text size
\baselineskip=15 pt plus .2 pt	% 3 lines/inch
\lineskip=3 pt			% minimum interline clearance
\raggedright			% lines are too short for good justification
\nopagenumbers			% the pagenumbers are in the heading
%
\def\heading#1{\noindent\line{\oneb Further Quantum Physics
\hfil #1\hfil\folio}\vskip2.5mm\hrule height1.0pt\vskip5mm}
\def\noi{\noindent}
\def\bra#1{\left< {#1} \right|}
\def\ket#1{\left| {#1} \right>}
\def\braket#1#2{\left< \left. {#1} \right| {#2} \right>}
\def\lesim{{\raise0.35ex\hbox{$<$}\kern-0.8em\lower0.75ex\hbox{$\sim$}}}
\def\i{{\rm i}}
\def\Vec#1{{\bf #1}}
\def\C2#1{{\cal #1}}
\def\V4#1{{\fam12 #1}}
\def\diff(#1,#2){{\partial #1\over\partial#2}}
\def\boxit#1{\vbox{\hrule\hbox{\vrule
\kern5pt\vbox{\kern5pt\hbox{#1}\kern5pt}\kern5pt\vrule}\hrule}}
\def\boxoneline#1{\medskip\noi\centerline{\boxit{$\displaystyle#1$}}\medskip}
\def\Eject{%
\ifodd\count0
   \ifodd\count99
      \eject
   \else
      \advancepageno
      \vskip2mm
      \hrule height0.3pt
      \vskip3mm
   \fi
\else
   \eject
\fi}
%
%\footline{\small CWPP \number\day/\number\month/\number\year\hfill}
%
%%%%%%%%%%%%%%%%%%%%%%%%%%%%%%%%%%%%%%%%%%%%%%%%%%%%%%%%%%%%%%%%%%%%%%%%%%%%%%%%
%
%  Viewgraph 1
%
\heading{SUMMARY OF QM}
\vfil
\centerline{\oneb FURTHER QUANTUM MECHANICS}
\bigskip
\centerline{\oneb Review of QM Concepts}
\bigskip
\bigskip\centerline{\oneb Dr C W P Palmer\hskip3em Hilary Term \number\year} 
\vfil\Eject
\heading{SUMMARY OF QM}
\centerline{{\bf Lecture 1:} QM Concepts}
\medskip\noi Quantum Mechanics is based on three new concepts, none of
which have simple correlates in classical physics: \medskip
\item{$\bullet$} The state, or ket $|\psi\rangle$;
\item{$\bullet$} The probability amplitude, or amplitude $\braket{\phi}{\psi}$.
\item{$\bullet$} The operator $\hat A$;
\medskip\noi
Combining the last two we obtain the final concept which derives from these:
\medskip
\item{$\bullet$}The matrix element $\langle\phi|\hat
A|\psi\rangle$.\vfil\Eject
\heading{SUMMARY OF QM}
\centerline{{\bf The ket} $|\psi\rangle$.}
\noi The ket is the complete quantum state of the system, from which we can
diagnose all its properties at a given time.
\smallskip\noi
\centerline {By contrast: \bf the complete Classical State}
The complete dynamical state of a {\it classical} system of particles consists of
the positions and momenta of all its particles (a point in {\it phase space}); or some other
set of variables equivalent to these. (Obviously this is not a complete
description --- we would need to add something about what particles were at each
of the specified positions. But it is a complete set of dynamical information.)

\noi Consider the planet Mercury, viewed as a point mass orbiting an
fixed attractive centre. We could specify its current position $\Vec r$, relative to a
defined coordinate system, and its current momentum $\Vec p$, or 
velocity $\Vec v$ --- a total of 6 variables. Or we could give the standard set
of orbital elements: (see http://ssd.jpl.nasa.gov/?ephemerides)
\vfil\noi
\centerline{\vbox{\halign{#\hfil\quad&#\hfil\qquad&#\hfil\quad&#\cr
$a$& Semi-major axis&$\Omega$&Longitude of Ascending Node\cr
$e$& Eccentricity   &$\omega$&Argument of Perihelion\cr
$i$& Inclination    &$t_p$   &Time of Perihelion Passage\cr}}}
\vfil\Eject
%\heading{SUMMARY OF QM}
%\medskip\line{\hfil\epsfxsize=110mm\epsffile{500px_Orbit1.svg.ps}\hfil}
%\vfil\Eject
\heading{SUMMARY OF QM}
\centerline{{\bf The ket} $|\psi\rangle$.}
\noi But this example is dynamically equivalent (same force law) to Hydrogen. The
complete quantum state for Hydrogen $\ket{n,\ell,m}$ is specified by just {\it three} numbers:
\medskip
\item{$n$} Principal quantum number ($E_n=-R/n^2$.)
\item{$\ell$} Angular Momentum quantum number ($L^2=\ell(\ell+1)\hbar^2$)
\item{$m$} Azimuthal quantum number ($L_z=m\hbar$).
\medskip\noi
In fact these specify an orbit shape and inclination but none of the
parameters in left-hand column.
\medskip
\centerline{\it The ket specifies the state less completely than the classical
state.}
\medskip\noi
But QM is right. So we should perhaps say
\medskip
\centerline{\it The classical state over-specifies the state.}
\medskip\noi
and if we recall the uncertainty principle this is obvious.
\vfil
\noi
But there is a choice about which if any of the classical variables are
defined by the ket . . .
\vfil\Eject
\heading{SUMMARY OF QM}
\centerline{{\bf The amplitude} $\braket{\phi}{\psi}$.}
\noi
Instead of the $\ket{n,\ell,m}$ states we could choose other complete sets of
kets:
\medskip\item{}$\ket{\Vec x}$: the particle is at $\Vec x$ (but momentum is
quite unknown);
\smallskip\item{}$\ket{\Vec p}$: the particle has momentum $\Vec p$ (but position
is quite unknown);
\smallskip\item{}$\ket{\Vec a(t)}$: a moving wave-packet ket, with a range
of positions around $\Vec a$ and a range of momenta around $m\dot\Vec a$;
\smallskip\item{}$\ldots$\medskip\noi
These specify some things {\it more} precisely than the $\ket{n,\ell,m}$ kets,
and some things less precisely. So the kets overlap in some sense. 
\smallskip\noi
$\braket{\phi}{\psi}$ denotes the probability amplitude that a particle whose
state is the ket $\ket{\psi}$ is found to be in the ket $\ket{\phi}$.
\vfil\Eject
\heading{SUMMARY OF QM}
\noi{\bf Aside on Probability and Probability Amplitude:}

\noi Probability (and probability density) are classical concepts. In all
cases they are used to answer incompletely specified problems --- ignorance of
some aspect of the system --- {\it epistemic} probability.

\noi Probability amplitude is a new concept invented for QM: a complex
number whose square is a a probability (or probability density). Why?

\noi Linearity of ket space --- amplitudes can add or cancel ---
interference, waves. You need this to be able to make all the various kets of
the previous page (how else can you combine different complete sets of 
information and make some information more precise and some less precise?).
\medskip\noi
So the desire to permit wave-like phenomena, like interference, has the effect
that the primary quantity of the theory is not directly accessible. --- measurable
things tend to involve squaring something.
\vfil\Eject
\heading{SUMMARY OF QM}
\noi{\bf The operator $\hat A$:} 

\noi Having introduced linearity, so that
kets are elements of a linear vector space (i.e. we can add them and multiply
by numbers) then it follows that we can have linear operators, and these turn
out to represent observables. Given an operator $\hat A$ we can find eigenkets
$$
\hat A\ket{a}=a\ket{a}.
$$
Hermitian operators play a special role because then several special things
happen:
\smallskip
\item{}$a$ is real
\item{} Eigenkets belonging to different eigenvalues are orthogonal
$\braket{b}{a}=0$.
\item{} The set of eigenkets is complete: $\displaystyle
\sum_a\ket{a}\bra{a}=1$.
\smallskip
Eigenkets of $\hat A$ represent states in which the observable $A$ has the
value $a$. 
\vfil\Eject
\heading{SUMMARY OF QM}
\noi{\bf The Measurement Postulate:} 

\noi
If we measure the observable $A$ on
a state $\ket{\psi}$ then:
\medskip
\item{(i)} The result is one of the eigenvalues of $A$, $a_n$
\smallskip\item{(ii)} with probability $|\braket{a_n}{\psi}|^2$.
\smallskip\item{(iii)}After the measurement the state is $\ket{a_n}$.
\medskip\noi
The first two parts simply follow from our definition of amplitude and
interpretation of eigenkets. So the measurement postulate defines the complete
probability distribution of possible outcomes $P(a_n)$. The expectation value
of the distribution is, as for any other probability distribution,
$\sum_aa\,P(a)$ and it's easy to show this is equal to $\bra{\psi}\hat
A\ket{\psi}$.  

\noi
The last part here is the {\it collapse of the wavefunction} and is a {\it big
deal.} No other part of QM has occasioned so much (or such bad-tempered)
dispute. This is the {\it measurement problem}. P\&P's take note.
\vfil\Eject
\heading{SUMMARY OF QM}
\noi{\bf The Matrix Element $\bra{m}\hat A\ket{n}$:}

\noi
Consider finding the expectation of $A$ in a superposition state
$\ket{\psi}=\sum_{n=1}^3c_n\ket{n}$. 

\noi (The basis $\ket{n}$ is unrelated to $A$;
e.g. $A$ is position and the basis states are energy eigenstates).

\noi
Then clearly we just substitute into both bra and ket:
$$
\bra{\psi}\hat A\ket{\psi}=\sum_{m=1}^3\sum_{n=1}^3c_m^*\bra{m}\hat
A\ket{n}c_n
$$
which we can conveniently write in a matrix and vector notation:
$$
\bra{\psi}\hat A\ket{\psi}=\Vec c^\dagger\Vec A\Vec c=
\pmatrix{c_1^*&c_2^*&c_3^*}\pmatrix{A_{11}&A_{12}&A_{12}\cr
A_{21}&A_{22}&A_{23}\cr
A_{31}&A_{32}&A_{33}}\pmatrix{c_1\cr c_2\cr c_3}
$$
where $A_{mn}=\bra{m}\hat A\ket{n}$. It's called a matrix element because it's
an element of a matrix! The matrix is the representative of the operator $\hat
A$ in this basis.

\noi The vector $\Vec c$ is the representative of the ket $\psi$ in the
basis $\ket{n}$. It's elements are $c_n=\braket{n}{\psi}$. (But the restriction
to just three states is just for this example --- it's an infinite vector
unless we have a reason to focus on just a subset of states.)
\vfil\Eject
\heading{SUMMARY OF QM}
\noi{\bf Dynamics:}

\noi States evolve in time --- so kets evolve in time:
$$
\i\hbar{d\hfil\over dt}\ket{\psi(t)}=H\ket{\psi(t)}
$$
where $H$ is the Hamiltonian operator (the energy expressed as a function of
position and momentum). 

\noi
This {\it looks} really simple if $H$ does not depend on
time: compare
$$
{df\over dt} + K f=0\quad\to\quad f(t)=f(0)\exp{(-Kt)}.
$$
Can we do the same here? Well we need to make the operator come before the ket:
$$
\ket{\psi(t)}=\exp{\left[-{\i Ht\over\hbar}\right]}\ket{\psi(0)}
$$
and we need to know how to take the exponential of an operator --- use the
series expansion:
$$
\exp{\left[-{\i Ht\over\hbar}\right]}=\sum_{n=0}^\infty\left({-\i
t\over\hbar}\right)^n{H^n\over n!}
$$
Since powers of $H$ are well defined, and we can add kets together, this isn't
mathematical nonsense! And in fact it solves the Schrodinger equation --- and
is the usual solution!
\vfil\Eject
\heading{SUMMARY OF QM}
\noi It looks particularly simple when we expand the ket at $t=0$ in
terms of the eigenkets of $H$: 
$$
\ket{\psi(0)}=\sum_ic_i\ket{E_i}\qquad\hbox{where} \quad
H\ket{E_i}=E_i\ket{E_i}.
$$
(These are time-independent kets. Sometimes time-dependent energy eigenkets 
are used, but I will always be explicit if I use them.) Then
$$
\exp{\left[-{\i Ht\over\hbar}\right]}\ket{E_i}=\sum_{n=0}^\infty\left({-\i
t\over\hbar}\right)^n{H^n\over n!}\ket{E_i}=\sum_{n=0}^\infty\left({-\i
t\over\hbar}\right)^n{E_i^n\over n!}\ket{E_i}=\exp(-\i E_it/\hbar)\ket{E_i}.
$$
So that our superposition state is just a sum of these
\boxoneline{
\ket{\psi(t)}=\sum_ic_i\exp(-\i E_it/\hbar)\ket{E_i}.
}
\noi The operator $U(t)=\exp{(-\i Ht/\hbar)}$ is called the {\it time development
operator} or the {\it propagator} because it propagates the ket forward in
time from $0$ to $t$.  It is also {\it unitary}: $U^{\dag}=U^{-1}$ which has the
important consequence that the normalisation is conserved:
$$
\braket{\psi(t)}{\psi(t)}=\bra{\psi(0)}U^{\dag}U\ket{\psi(0)}=\braket{\psi(0)}{\psi(0)}.
$$
\vfil\Eject
\heading{SUMMARY OF QM}
\noi{\bf Example: Particle in a box}
We consider a box of width $a$, from $0$ to $a$. Then the eigenfunctions and
eigenvalues are easily found:
$$
E_n={n^2 \pi^2\hbar^2\over 2 m a^2}\qquad u_n(x)=\braket{x}{n}=\sqrt{2\over
a}\sin\left[{n\pi x\over a}\right].
$$
(You may have see a version of this with a different definition of $a$, and
symmetry about $x=0$; this version has the nice property that we have a single
formula for all the eigenfunctions).

To see wavepackets based on this see: (linked from FQM webpage)

http://www.physics.ox.ac.uk/users/palmerc/tdse\_Applet.htm
\vfil
\noi{\bf Correspondence Principle:} One implication of this follows from
the time-dependence that we get in expectation values:
$$
{dE_n\over dn}\to h\nu_{\rm class}(E_n)\qquad\hbox{at large $n$}
$$
where $\nu_{\rm class}$ is the classical orbit frequency as a function of energy.
\vfil\Eject
\heading{SUMMARY OF QM}
\centerline{\bf Commutators}
\noi One of the most striking non-classical features of QM is the
non-commutation of operators representing classical variables. What does it
mean?
\smallskip
$[\hat A,\hat B]=0$: There is a really powerful theorem, which we will use
repeatedly, which states that if two operators commute then there exists a
complete set of mutual eigenkets $\ket{a,b}$:

\leftline{\hskip4em$\hat A\ket{a,b}=a\ket{a,b}$}
\leftline{\hskip4em$\hat B\ket{a,b}=b\ket{a,b}$.}\noi
This also implies that the observables $A$ and $B$ can be measured compatibly.

\noi(There is a slight wrinkle concerning degeneracy).
\medskip
$[\hat H,\hat A]=0$: In the case of operators commuting with the Hamiltonian then there
is a further consequence: for any state $\ket{\psi(t)}$, the expectation value
of $A$ is independent of time
$$
{d\hfil\over dt}\bra{\psi(t)}\hat A\ket{\psi(t)}=0
$$
and in a borrowing of classical terminology $A$ is often referred to as a
constant of the motion. \smallskip\noi
Note that this is true for all $\hat A$ commuting with $\hat H$
even if they don't commute with each other. $[\hat H,\hat L_i]=0$ for all
components of $\hat\Vec L$, so all are constants of the motion.  
\vfil\Eject
\heading{SUMMARY OF QM}
\centerline{\bf Commutators}
\noi (Finally there are commutators of the form $[a,\hat B]=c\,a$ for
some real number $c$. This
implies that $a$ is a ladder operator: but it can't be a classical observable!
That's because if $\hat A$ and $\hat B$ are Hermitian then $[\hat A,\hat B]$
is anti-Hermitian: that is, a Hermitian operator multipled by $\i$. The
absence of an $\i$ on the RHS means that $a$ is neither Hermitian nor
anti-Hermitian.) 
\vfil
\centerline{\bf Scaling the Hamiltonian}
\noi
The harmonic oscillator Hamiltonian in the $x$-representation contains three
constants:
$$
H=-{\hbar^2\over 2m}{d^2\hfil\over dx^2}+{1\over
2}m\omega^2x^2\quad\hbox{involves}\quad\hbar,\>m,\>\omega.
$$This means we can devise a system of units specially adapted for the
oscillator:
\medskip\noi\centerline{\vbox{\halign{\quad\hfil#\hfil\qquad&\hfil#\hfil\quad\cr
Constant&Dimensions\cr\noalign{\vskip3pt\hrule\vskip3pt}
$m$&$[M]$\cr
$\omega$&$[T^{-1}]$\cr
$\hbar$&$[ML^2T^{-1}]$\cr\noalign{\vskip3pt\hrule}
}}}
\noi Thus $m$ has dimensions of mass, $\omega^{-1}$ of time and
$\sqrt{\hbar/m\omega}$ of length, and in these units the unit of energy is $\hbar\omega$.
\vfil\Eject
\heading{SUMMARY OF QM}
\noi
We now switch to these units --- we define 
$\displaystyle
x=\sqrt{\hbar\over m\omega}\,\V4x\qquad E=\hbar\omega \V4E
$
which gives the time-independent Schrodinger equation as
$$
{\hbar^2\over 2m}\,{m\omega\over\hbar}{d^2\psi\over d\V4x^2}
+{1\over2}m\omega^2{\hbar\over m\omega}\V4x^2\psi=\hbar\omega\V4E\psi.
$$
\noi$\hbar\omega$ cancels throughout to leave 
$\displaystyle\qquad
-{d^2\psi\over d\V4x^2}+\V4x^2\psi=2\V4E\psi
$

\noi
which is the underlying equation stripped of constants.

\noi
In these units  the eigenvalues are just $\V4E_n=n+{1\over2}$, 

the ground
state wavefunction is
$\displaystyle\psi_0(\V4x)={e^{-\V4x^2/2}\over\pi^{1/4}}$ 

and the creation operator is $\displaystyle
a^\dagger={1\over\sqrt2}\,(\V4x-\i\V4p)$ where $\V4p=-\i(d/d\V4x)$.

\noi We can always recover the wavefuction in standard units by replacing
$\V4x$ with $\sqrt{m\omega/\hbar}\,x$ and multiplying the wavefuntion by 
$(m\omega/\hbar)^{1/4}$ because $\psi$ has dimensions of $[L]^{-1/2}$:
$$
\psi(x)=\left({m\omega\over\pi\hbar}\right)^{1/4}\,\exp{(-m\omega x^2/2\hbar)}.
$$
\vfil\Eject
\bye
