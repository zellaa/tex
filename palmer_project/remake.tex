\documentclass{article}
%%%% page geometry
\usepackage{remake}
%\usepackage[a4paper,left=3.5cm,right=3.5cm,top=\dimexpr15mm+1.5\baselineskip,bottom=3cm,landscape]{geometry}
%%%% general maths and physics tools
\usepackage{amssymb,mathtools}
\usepackage{physics}

\usepackage[dmyyyy]{datetime}

%%%% font formatting
%\usepackage[12pt]{moresize}
%\usepackage[protrusion=true]{microtype}
%\usepackage[utf8]{inputenc}
%%%%% use a slightly thicker font
%\DeclareFontShape{OT1}{cmr}{mx}{n}%
     %{<->cmr10}{}

%%%% header and footer
%\usepackage{fancyhdr}
%\renewcommand{\headrulewidth}{2pt} % header line width
%\pagestyle{fancy}
%\fancyhf{}
%\fancyhfoffset[L]{1cm} % left extra length
%\fancyhfoffset[R]{1cm} % right extra length
%\lhead{\bf \huge Frrther Quantum Physics \hfill SUARY OF QM \hfill \thepage\vskip -8mm}
%\lfoot{ \Large ZBB \today}

%%%% setup boxes; we use the empheq environment
\usepackage{empheq}
%%%% set the initial box margins
\setlength{\fboxsep}{3\fboxsep}
%%%% set line weight
\setlength\fboxrule{1pt}

%%%% define lbox and cbox; increase the argument for hspace to increase horizontal padding.
\newcommand*\lbox[1]{\boxed{\hspace{0em}#1\hspace{0em}}}
\newenvironment{cbox}{%
    \empheq[box=\lbox]{align*}}{\endempheq}

%%%% comment these next 2 out to get landscape A4 lecture slides; 
%%%% uncomment to get 2*A5 landscape handouts on a single page

%\usepackage{pgfpages} 
%\pgfpagesuselayout{2 on 1}[a4paper]

%%%% CUSTOM COMMANDS





%\usepackage{lipsum}
%\usepackage{tikz}
%\usepackage{anyfontsize}

%\usetikzlibrary{positioning,calc}

%\makeatletter
%\renewcommand{\maketitle}{%
    %\begin{titlepage}
    %\vspace*{\fill}
	%\begin{center}%
	%\fontsize{21pt}{10pt}\selectfont
	%\textsc{\bf\@title\vspace{2\bigskipamount}}\\[4ex]
	%\@subtitle\\[4ex]
	%%{\@title}
	%\end{center}
    %\vspace*{\fill}
    %\end{titlepage}
	%}
%\makeatother
	\usepackage{remake}
\author{Zella Baig}
\term{Hilary Term 2021}
\title{further quantum mechanics}
\subtitle{Review of QM Concepts}
\course{Further Quantum Physics}
\topic{summary of QM}


%%%% Last edited 9th Feb 2021
\begin{document}
%%%% create the title page
%\vspace*{\fill}
%\begin{center}
    %\fontsize{21pt}{10pt}\selectfont
	%\textsc{\bf{FURTHER QUANTUM MECHANICS \vspace{2\bigskipamount} \\
	%Review of QM Concepts \vspace{4\bigskipamount} \\
	%Zella Baig \hspace{30mm} Copied from Dr Palmer}}
%\end{center}
%\vspace*{\fill}
\maketitle
\newpage

%%%% set font
\fontseries{mx}\selectfont
\LARGE
%%%% the start of the slides proper:
\begin{center}
	\textbf{Lecture 1:} Proof of Concept
\end{center}
Quantum Mechanics is based on three new concepts, none of which have simple correlates in classical physics:
\begin{itemize}
	\item The state, or ket $\ket{\psi}$;
	\item The probability amplitude, or amplitude $\braket{\phi}{\psi}$;
	\item The operator $\hat A$;
\end{itemize}
Combining the last two we obtain the final concept which derives from these:
\begin{itemize}
	\item The matrix element $\mel{\phi}{\hat A}{\psi}$.
\end{itemize}
\newpage
\begin{center}
	\textbf{The ket} $\ket{\psi}$.
\end{center}
The ket is the complete quantum state of the system, from which we can diagnose all its properties at a given time.
\begin{center}
	By contrast: \textbf{the complete Classical State}
\end{center}
Here is an example of a boxed equation, using the 'cbox' environment.
\begin{cbox}
		\ket{\psi(t)}=\sum_i c_i\exp(-i E_it/\hbar)\ket{E_i}.
\end{cbox}
'cbox' is a wrapper for the align* environment; so you may have line breaks:
\begin{cbox}
	\ket{\psi(t)}&=\sum_ic_i\exp(-i E_it/\hbar)\ket{E_i}.\\
	\ket{\phi(t)}&=\sum_ic_i\exp(-i E_it/\hbar)\ket{E_i}.
\end{cbox}
We may also use inline boxes with the 'lbox' command: see, for example this inline box: \lbox{y=mx+c} as well as another one: \lbox{y \neq mx+b}
\newpage
\begin{center}
	\textbf{Newpage} JUST A TEST
\end{center}
The ket is the complete quantum state of the system, from which we can diagnose all its properties at a given time.
\begin{center}
	By contrast: \textbf{the complete Classical State}
\end{center}
Here is an example of a boxed equation, using the 'cbox' environment.
\begin{cbox}
		\ket{\psi(t)}=\sum_ic_i\exp(-i E_it/\hbar)\ket{E_i}.
\end{cbox}
cbox is a wrapper for the align* environment; so you may have line breaks:
\begin{cbox}
	\ket{\psi(t)}&=\sum_ic_i\exp(-i E_it/\hbar)\ket{E_i}.\\
	\ket{\phi(t)}&=\sum_ic_i\exp(-i E_it/\hbar)\ket{E_i}.\\
	\ket{\phi(t)}&=\sum_ic_i\exp(-i E_it/\hbar)\ket{E_i}.
\end{cbox}
One may also use inline boxes with the 'lbox' command: see, for example this inline box: \lbox{y=mx+c} as well as another one: \lbox{y \neq mx+b}
\end{document}
