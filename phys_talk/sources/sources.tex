\documentclass{article}
\begin{document}
\begin{thebibliography}{99}
	\bibitem{school}
	D. Gooding, “’In Nature’s School’: Faraday as an Experimentalist,” in D. Gooding and F. James, \emph{Faraday Rediscovered, Essays on the Life and Work of Michael Faraday}, (Springer, 1984), pp. 105-136

\bibitem{finalsteps}
	D. Gooding, “Final Steps to the Field Theory: Faraday's Study of Magnetic Phenomena, 1845-1850.”,  \emph{Historical Studies in the  Physical Sciences}, vol. 11 (1981), pp. 231-275.
\bibitem{rilect} 
	Royal Institute Manuscript, F4 J12, p. 17

\bibitem{maxwellians}
	B. Hunt, ``The Maxwellians'', (Ithaca, 1991)

\bibitem{diary1845}
	M. Faraday, ``Diary'', vol. 4, 13 September 1845

\bibitem{farline1}
J. Maxwell, “On Faraday's Lines of Force”, in \emph{Transactions of the Cambridge Philosophical Society}, Cambridge, 1856, pp. 155-159

\bibitem{farline2}
	J. Maxwell, “On Faraday's Lines of Force”, in \emph{Transactions of the Cambridge Philosophical Society}, Cambridge, 1856, pp. 155-156

\bibitem{thomlet}
W. Thomson and S.P. Thompson, ``Letter from Thomson to Faraday, 19 June 1849'', \emph{The Ife of William Thomson Lord Kelvin of Largs}, London, 1910, pp. 214-216

\bibitem{masters}
	A. C. Warwick, ``Masters of Theory: Cambridge and the Rise of Mathematical Physics'', (University of Chicago Press, 2003).

\bibitem{field}
	D. Gooding, “Faraday, Thomson, and the concept of the magnetic field,” Brit. J. Hist. Sci., 13 (1980), pp. 91-120

\bibitem{siegel}
	D. Siegel, ``Innovation in Maxwell's Electromagnetic Theory: Molecular Vortices, Displacement Current, and Light'', (Cambridge, 1992)

	\bibitem{leeuwenhoekgen} 
	I. Davis, ``Antoni van Leeuwenhoek and measuring the invisible: The context of 16th and 17th century micrometry'', \emph{Studies in History and Philosophy of Science Part A}, Volume 83, 2020, Pages 75-85

\bibitem{sperm}
	R. Kempers, ``The Tricentennial of the Discovery of Sperm'', \emph{Fertility and Sterility}, Volume 27, Issue 5, 1976, Pages 603-605

\bibitem{camgal}
	N. Swerdlow, “Galileo’s telescopic discoveries and their evidence for the Copernican theory,” in P. Machamer, ed., \emph{The Cambridge Companion to Galileo}, Cambridge, 1998

\bibitem{artnat}
B. Bensaude-Vincent and W. Newman eds, \emph{The Artificial and the Natural: an evolving Polarity}, Ca. Mass., 2007

\bibitem{inst}
T. Hankins and R. Silverman, \emph{Instruments and the Imagination}, Princeton, 1995

\bibitem{bacon}
S. Gaukroger, \emph{Francis Bacon and the Transformation of Philosophy}, Cambridge, 2002

\bibitem{excel}
P. Dear, \emph{Discipline \& Experience. The Mathematical Way in the Scientific Revolution}, Chic., 1995

\bibitem{newt1} 
	I. Newton and D.T. Whiteside, ``The Mathematical Papers of Isaac Newton'', vol. 8, (Cambridge, 1967-1981), pp. 287-289

\bibitem{newt2}
	I. Newton and D.T. Whiteside, ``The Mathematical Papers of Isaac Newton'', vol. 7, (Cambridge, 1967-1981), pp. 291

\bibitem{newt3}
	I. Newton, MS Add. 3970, f. 243r

\bibitem{corr}
	I. Newton, H. W. Turnbull, J. F. Scott, A. R. Hall, and L. Tilling, ``The Correspondence of Isaac Newton'', (Cambridge, 1959-77), vol. 5,  pp. 398-399

\bibitem{westfall}
	R. S. Westfall, ``Never at Rest'', (Cambridge, 1984), ch. 10

\bibitem{shapiro}
	A. Shapiro, "Newton's "Experimental Philosophy"." Early Science and Medicine 9, no. 3 (2004), pp. 185-217

\bibitem{gucci}
	N. Guicciardini, ``Isaac Newton on mathematical Certainty and Method'', (London, 2009) 

		\bibitem{collpap}
R. Schulmann, et al., ed., The Collected Papers of Albert Einstein, Vol. 8: The Berlin Years. Correspondence: 1914–1918, (Princeton: Princeton University Press, 1998), doc. 43.
\bibitem{brown54}
H. R. Brown, Physical Relativity: Space-time Structure from a Dynamical Perspective, Oxford University Press, 2005. Paperback revised edition October 2007 p. 54
\bibitem{lum}
H. Poincaré, Electricité et Optique. I. Les théories de Maxwell la Théorie Électromagnétique de la Lumière  (Sorbonne lectures of spring 1888, edited by J. Blondin; Paris,1890) pp. IX, XIV
\bibitem{ponher}
HP. Poincaré, Electricité et Optique. La lumière et les Théories électrodynamiques (Sorbonne lectures of spring 1888, 1890, and 1899 edited by J. Blondin E. Neculcea; Paris, 1901) pp. 403-420
\bibitem{paty}
Cf. Paty, Einstein Philosophe, op. cit., note 1, pp. 264-271.
\bibitem{measure}
H. Poincaré. "La mesure du temps." Revue De Métaphysique Et De Morale 6, no. 1 (1898): pp. 1-13.
\bibitem{reaction}
H. Poincaré, La Théorie de Lorentz et le principe de réaction, Archive néerlandaises des sciences exactes et naturelles, no. 5, (1900), pp. 252–278
\bibitem{op32}
M. Janssen and J. Stachel. ‘The optics and electrodynamics of moving bodies’, Volume 265 of Preprint, Max-Planck-Institut für Wissenschaftsgeschichte (2004) p. 32
\bibitem{rindler}
W. Rindler. ‘Einstein’s priority in recognizing time dilation physically’, American Journal of Physics, 38: pp. 1111–1115 (1970)
\bibitem{fond}
H. Poincaré and G. Halsted, The Foundations of Science: Science and Hypothesis, the Value of Science, Science and Method, (New York, 1929). p. 147
\bibitem{jansen}
M. Janssen, ‘Reconsidering a scientific revolution: The case of Einstein versus Lorentz’, Physics in Perspective, 4: 421–46 (2002)
\bibitem{russ}
R. McCormmach, (1970). H. A. Lorentz and the Electromagnetic View of Nature. Isis 61 (4):459-497.
\bibitem{op}
 M.Janssen and J. Stachel. ‘The optics and electrodynamics of moving bodies’, Volume 265 of Preprint, Max-Planck-Institut für Wissenschaftsgeschichte (2004)
\bibitem{bro}
H. R. Brown, Physical Relativity: Space-time Structure from a Dynamical Perspective, Oxford University Press, 2005. Paperback revised edition October 2007
\bibitem{poinc}
O. Darrigol, (1995). Henri Poincaré's criticism of Fin De Siècle electrodynamics. Studies in History and Philosophy of Science Part B: Studies in History and Philosophy of Modern Physics 26 (1):1-44.

\end{thebibliography}
\end{document}
